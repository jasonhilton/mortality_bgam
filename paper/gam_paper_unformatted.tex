\documentclass[10pt,a4paper]{article}
\usepackage[utf8]{inputenc}
\usepackage{amsmath}
\usepackage{amsfonts}
\usepackage{amssymb}

\usepackage[a4paper]{geometry}
\usepackage{graphicx}
\usepackage[textwidth=8em,textsize=small]{todonotes}
\usepackage{amsmath}
\usepackage{natbib}


\addtolength{\oddsidemargin}{-.50in}
\addtolength{\evensidemargin}{-.50in}
\addtolength{\textwidth}{1.0in}
\addtolength{\topmargin}{-.50in}
\addtolength{\textheight}{1.0in}


\title{Projecting UK Mortality using Bayesian Generalised Additive Models}
\author{Jason Hilton, Erengul Dodd, Jon Forster, Peter W. F. Smith\\
Centre for Population Change,\\
University of Southampton,\\
Southampton,\\
United Kingdom,\\
SO17 1BJ \\
J.D.Hilton@soton.ac.uk
}

\begin{document}
\maketitle
\begin{abstract}
Forecasts of mortality provide vital information about future populations, with implications for  pension and health-care policy as well as for decisions made by private companies about life insurance and annuity pricing. Stochastic mortality forecasts allow the uncertainty in mortality predictions to be taken into consideration when making policy decisions and setting product prices. Longer lifespans imply that forecasts of mortality at ages 90 and above will become more important in such calculations.

This paper presents a Bayesian approach to the forecasting of mortality that jointly estimates a Generalised Additive Model (GAM) for mortality for the majority of the age-range and a parametric model for older ages where the data are sparser. The GAM allows smooth components to be estimated for age, cohort and age-specific improvement rates, together with a non-smoothed period effect.  Forecasts for the United Kingdom are produced using data from the Human Mortality Database spanning the period 1961--2013. A metric that approximates predictive accuracy under Leave-One-Out cross-validation is used to estimate weights for the `stacking' of forecasts with different points of transition between the GAM and parametric elements.

Mortality for males and females are estimated separately at first, but a joint model allows the asymptotic limit of mortality at old ages to be shared between sexes, and furthermore provides for forecasts accounting for correlations in period innovations. The joint and single sex model forecasts estimated using data from 1961--2003 are compared against observed data from 2004-2013 to facilitate model assessment.
\end{abstract}

\section*{Acknowledgments}\label{acknowledgments}

This work was supported by the ESRC Centre for Population Change - phase II (grant ES/K007394/1), and a research contract (“Review of Mortality Projections”) between the Office of National Statistics and the University of Southampton. The use of the IRIDIS High Performance Computing Facility, and associated support services at the University of Southampton, in the completion of this work is also acknowledged. Earlier work on this model was presented at a joint Eurostat/UNECE work session on demographic projections \citep{Forster2016}. All the views presented in this paper are those of the authors only.

\include{gam_mortality}


\bibliography{GAM}
\bibliographystyle{kluwer}

\end{document}